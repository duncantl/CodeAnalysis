\section{Motivation}
One of the primary strengths of the \R language and computing environment that makes it very
different from other languages such as \C/\Cpp and \Java is its interactivity.
Rather than writing complete programs or even functions, we often write one command, examine the
results, and decide what do next. We build scripts incrementally.  We often use this style to
develop functions that we reuse, adapting and extending them incrementally in a new context.
This somewhat \textit{ad hoc} nature is quite different from fromal software engineering
practices where often teams develop requirements and specifications and unit tests before
implementing the code.  This evolutionary development process means that the resulting
code may not have the same global perspective, be as succinct as it could be,
and may contain errors such as undefined variables or extra unnecessary commmands that were added
from the experimental context. We leave code in a script in case we need it in the future;
we cut and paste code from a script into a function and forget to change the variable names.




When we write \R scripts to do data manipulation and/or analysis
