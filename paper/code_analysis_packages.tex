\section{Prior work on R Code Analysis Packages}
\label{sec:code_analysis_packages}

Several R packages analyze R code. This
section provides an overview of prior work in the area.
Table~\ref{table-codeanalysis} compares existing packages.

\begin{table}[]
    \centering
    \label{table-codeanalysis}
    \begin{tabular}{ll}
        \textbf{Package}    & \textbf{Use Case}
        \\ CodeDepends  & infer higher level semantics
        \\ codetools    & byte compilation, general tools
        \\ covr         & measure unit test coverage
        \\ globals      & identify global variables
        \\ lintr        & identify problems in style and usage
    \end{tabular}
    \caption{Popular R packages for code analysis}
\end{table}

There's significant redundancy in the functionality among these packages.
For example, most of them have code to determine if an object is
defined locally. This is a problem, because they don't all correctly handle the
corner cases of \texttt{<<-, ->>, assign()}.

\subsection{Packages}

\textbf{codetools} ships with R as a recommended package
\cite{R-codetools}. It supports byte compilation. It includes high level
functions such as \texttt{checkUsage} to inspect closures for possible
problems, and low level functions such as \texttt{walkCode} to traverse an
AST.

\textbf{CodeDepends} gathers all kinds of information on code
\cite{R-CodeDepends}. This includes information on files accessed,
attaching packages, variables used / redefined / updated / removed,
and functions called. It provides a user facing way to handle functions
with nonstandard evaluation.

\textbf{globals} statically detects which variables are needed to evaluate
an expression, and then uses R's normal rules to dynamically search
environments and find them \cite{R-globals}. This package supports
exporting these variables to distributed compute environments in the future
package.

\textbf{covr} checks unit test coverage of R code \cite{R-covr}. It is a specialized code
analysis tool with wide applicability. Several hundred R packages on CRAN
list it as a suggested package. From the vignette: ``Function definitions
are modified by parsing the abstract syntax tree and inserting trace
statements.'' Looking at the source code it walks the AST and replaces
\texttt{CODE} with the following:

\begin{lstlisting}
{
    count(key)
    CODE
}
\end{lstlisting}

It resembles \texttt{base::trace}, because both replace actual functions
with instrumented versions. It handles control flow functions \texttt{if,
for, while, switch} specially so that it can descend into all branches. It
relies on a third party tool to check coverage of C code.

\textbf{lintr} checks R code for style and semantic issues \cite{R-lintr}.
Users can customize options or add their own checks. The package uses
\texttt{utils::getParseData()} to examine code at the token and positional
level. Identifying the positions of issues in a physical source file allows
an integrated development environment (IDE) to point out these issues to
the developer. It does not fix the issues it finds. This is too bad,
because many of the style issues are easy to fix.

The analysis uses at least four different
representations of the code:

\begin{enumerate}

    \item \textbf{AST} The vignette says that new linters will be called on
        each top level expression in the file. This seems like the most
        natural way to do things, although less convenient for identifying
        positions.
    \item \textbf{Tokens} These seem generally more complex
        compared to the AST. \texttt{unneeded\_concatenation\_linter}
        seems to go to a lot of trouble to match delimiters, which doesn't
        come up in the AST.
    \item \textbf{XML} \texttt{seq\_linter} represents
        the code in XML and then uses xpath to detect certain patterns.
    \item \textbf{functions} \texttt{object\_usage\_linter} evaluates
        code and creates the function objects to pass off the work to
        the codetools package. It seems strange to be actually evaluating
        code in the context of static analysis.

\end{enumerate}

This package focuses mainly on style. Of the 18 default linters in version
1.0.2.9000, 11 deal with cosmetic things such as whitespace and naming.
The following linters may indicate actual issues in the code:
\texttt{
T\_and\_F\_symbol\_linter,
undesirable\_operator\_linter,
absolute\_path\_linter,
nonportable\_path\_linter,
seq\_linter,
extraction\_operator\_linter,
object\_usage\_linter
}
